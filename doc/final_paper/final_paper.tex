\documentclass[11pt]{article}

\usepackage{times}
\usepackage[utf8]{inputenc} % allow utf-8 input
\usepackage[T1]{fontenc}    % use 8-bit T1 fonts
\usepackage{url}            % simple URL typesetting
\usepackage{graphics}
\usepackage{color}
\usepackage{amsfonts}       % blackboard math symbols
\usepackage{amsmath}       % blackboard math symbols
\usepackage{amssymb}
\usepackage{multicol}
\usepackage{enumerate, multirow, color, graphicx, lastpage, listings, tikz, pdflscape, subfigure, float, polynom, tabularx, forloop, geometry, listings, fancybox, tikz, forest, tabstackengine, cancel}

\graphicspath{{../graphics/}}
\usepackage{geometry}
\geometry{left=2.8cm,right=2.8cm,top=2.6cm,bottom=2.6cm}
\usepackage{fancyhdr}
\pagestyle{fancy}
\usepackage{hyperref}% should be the last package you include

\newcommand{\theteam}{}
\newcommand{\team}[1]{\def\theteam{#1}}


\fancyhead[L]{\rightmark} % I changed this from : \fancyhead[L]{\theteam} (think it is just less ugly like that)
\fancyhead[R]{\thepage}
\cfoot{}

\setlength{\parindent}{0pt}

% For tighter image bounds and caption bounds
\setlength{\belowcaptionskip}{-10pt}
\setlength{\abovecaptionskip}{-10pt}
\setlength\intextsep{0pt}

\team{Jessica Bader, 5624582; Philipp Noel von Bachmann, 4116220 }
\title{Extremely Lossy Compression through Reinforcement Learning}
\author{\theteam}

\DeclareMathOperator*{\argmax}{arg\,max}

\begin{document}
\maketitle

\begin{multicols*}{2}
\section{Introduction}\label{sec:Introduction}
Image compression exists in multiple forms. Lossless compression mandates
original data be perfectly reconstructed, but is limited by data entropy. To
compress further, information content must be sacrificed. Therefore, lossy
compression minimizes the loss for a specific bitrate. But which content should
be discarded? Early image compression loss-functions, like $L_2$-loss, give each
pixel equal value. However, some parts retain more valuable information than
others. Research has hand-crafted improved metrics, like MS-SSIM
~\cite{1292216}, but defining these is task-specific. Ultimately, achieving the
most extreme forms of compression requires task-dependent methods to quantify
information value, preserving only the minimum required to complete the task.
However, general methods for this process would be useful. This facilitates a
natural transition into the reinforcement learning (RL) space, where agents
already complete tasks by acting on an environment to maximize rewards. The goal
of this research is to define a method through which data is compressed such
that the agent can maximize the reward in the reconstructed environment.

\section{Related Works}
Balle et. al's work in modifying neural networks (NNs) (variational autoencoders) for data compression provided the inspiration for modifications to the RL agent for data compression ~\cite{DBLP:journals/corr/BalleLS16a}.\\
Furthermore, a variety of work explores using RL agents in data compression, including pruning for image compression~\cite{pruning2}~\cite{pruning3}~\cite{pruning1}, optimal codebook mapping~\cite{8954232}, NN architecture compression~\cite{nn_arch_comp}, sentence compression~\cite{inbook}, and state compression for RL agents~\cite{9387144}. However, to the best of our knowledge, no previous research exists in the task-dependent image compression domain.
\section{Methods}

\subsection{Architecture}\label{methods:Architecture}
    In Deep-Learning based compression, an encoder NN embeds data in a latent space, where it is losslessly encoded and transmitted. The decoder NN decompresses the latent space and returns to the input domain.

    \begin{figure}[H]
        \centering
        \includegraphics[width=\linewidth]{images/architecture.pdf}
        \caption{Overall architecture}
        \label{fig:Architecture}
    \end{figure}

    In our approach, a RL agent acts as the first
    NN. The agent contains a feature extractor, which
    extracts information to determine
    the action. Ideally, this content should also be important
    for reconstruction. Then the latent space is losslessly encoded and
    transmitted with an algorithm of choice, such as ANS (for training this
    step is unnecessary as the latents will be losslessly decompressed). After decoding, the latents they are passed through a normal upsampling NN.
    Figure \ref{fig:Architecture} shows an overview of the architecture.

\subsection{Theory}
    Lossy compression balances a low bitrate and a low distortion.
    This section describes how these objectives apply to our project.

    \subsubsection{Bitrate}
        Determining algorithm bitrate requires observing the encoding
        process, where the input image is mapped to the
        latent space, see \ref{methods:Architecture}. This decreases the
        image bitrate if the latent entropy is lower than the image entropy and a sufficient encoding distribution for latent transmission is chosen. If the latent
        space distribution of $z$ is given by $Q_\theta(Z=z \vert x)$ where $\theta$
        parametrizes the distribution and $P(Z=z)$ is used to encode the latents for transmission, the bitrate is given by the cross entropy
        \begin{equation}\label{eq:BitRate}
            \mathbb{E}_{z \sim Q_\theta(Z=z \vert x)}[log(P(Z=z))]
        \end{equation}
        In addition, $Q=P$ will give the perfect encoding distribution. Hence, the
        entropy is a lower bound on the cross-entropy.

        Because NNs have real valued outputs, $Q_\theta(Z=z
        \vert x)$ is a continuous PDF, resulting in a high bit rate. Therefore, one
        approach is to round the latents \cite{DBLP:journals/corr/BalleLS16a}. This reduces $Q$ to a discrete
        probability distribution, which reduces the cross entropy. However, rounding
        is undesirable during training, since its gradient is 0 nearly everywhere. Therefore, we replace
        rounding during training with adding uniform noise $u$, see also
        figure \ref{fig:Architecture}. This shifts the values
        similarly to rounding, but is differentiable.

        Another effect of adding uniform noise is that we can reparametrize equation
        \ref{eq:BitRate} and take the expectation over the uniform noise:
        \begin{equation}
            \mathbb{E}_{u \sim U[-\epsilon, \epsilon]}[log(P(Z=\hat{z} + u))]
        \end{equation}
        where $\hat{z}$ is the mean of $Q_\theta(Z=z \vert x)$. \newline

        % (To be more precise: We start with a continuous pdf. Rounding creates a
        % bunch of delta pulses which are nondifferentiable. However we can
        % approximate the continuous pdf by adding uniform noise to the new discrete
        % pdf. Therefore we can also just add noise instead of round?)


        % In turn, this means that the encoding network shouldn't encode information
        % in small differences. One approach to achieve this behaviours is to add
        % noise to the latent values, which forces the RL-agent to get more robust to
        % noise.

        % [TODO: this part is just disregared in balle]
        The encoding distribution $P$ is naively chosen as Normal. The
        mean can be chosen arbitrarily given a powerful enough transformation, so it is
        set to 0 for simplicity. The variance is learned during the training process.

    \subsubsection{Distortion}\label{sub:Distortion}
        Bitrate decrease is often traded for distortion increase.
        Therefore, we need a performance measure to evaluate reconstructed image distortion $\hat{x}$. A traditional metric is the Mean Squared Error (MSE),
        \begin{equation}\label{equ:L2}
            \mathbb{E}_{x, \hat{x}}[\sum_{i,j} (x_{ij} - \hat{x}_{ij})^2]
        \end{equation}
        Early experimentation (see section 4.1) found that some task-important details, specifically ball location, were omitted as they accumulated little MSE penalty; attempting to
        recover these aspects, we devised a second loss scheme, referred to as
        latent loss. The motivation was to reward the decoder for reconstructing the
        image such that the same features would be extracted, resulting in
        important aspect preservation. For this loss, we passed the
        reconstructed image through the feature extractor and evaluated the MSE
        between the original and reconstructed latent space.


        % If humans look at images however, they often just care about specific types
        % of distortion. A slightly lower brightness for example wouldn't matter for
        % most images, but probably a different colour or more generally they care
        % more about semantic distortion that would give them a different
        % interpretation and representation of the image. MSE however gives these
        % distortions all the same values.

        % As the RL-Agent is used to encode the data and therefore should extract the
        % relevant features, it seems reasonable to let the agent also judge the
        % reconstruction. Therefore, a second approach will be to compare not the
        % original image and the reconstuction, but rather the agents representation
        % of the images to measure similarity. In practive, this means passing both
        % images through the feature extractor and measuring the L2 distance in latent
        % space.


        % \subsection{Decoder loss}\label{sub:Decoder_Loss}
        %     For our decoder, we tried several loss functions. The simplest version was
        %     the MSE between the original and reconstructed images. \\
        %     In early experimentation (see section 4.1), we found that some details which
        %     may be considered important to the task, specifically the location of the
        %     ball, were omitted as they accumulated little MSE penalty; in an attempt to
        %     recover these aspects, we devised a second loss scheme, which we refer to as
        %     latent loss. The motivation was to reward the decoder for reconstructing the
        %     image in such a way that the same features would be extracted, resulting in
        %     preserving the most important aspects. For this loss, we passed the
        %     reconstructed image through the feature extractor and evaluated the MSE
        %     between the original latent space and the reconstructed latent space.





    % \subsection{Objective function from view of variational inference}
    %     In Lossy image compression, the objective function is given by the ELBO:
    %     \begin{align}
    %         & \mathbb{E}_{z \sim Q(Z= z)}[log P(X, Z= z) - log Q(Z = z)]\\
    %         & = \mathbb{E}_{z \sim Q(Z= z)}[log P(X \vert Z= z) - log \frac{Q(Z = z)}{P(Z = z)}]\\
    %         & = \mathbb{E}_{z \sim Q(Z= z)}[log P(X \vert Z= z)] - D_{KL}[Q(Z = z)\Vert P(Z = z)]\\
    %     \end{align}

    %     Problem is, that we cannot differenciate by q since expectation is over q,
    %     so just fix q by uniform distribution, and do reparametrization trick

    %     now entropy of q becomes fixed so we can remove from optimization, just need
    %     to care about $E_U[P(Z + U)]$

    %     just need to choose encoding distribution, choose naively as normal with
    %     mean 0 and learned variance since needs to be flexible.

    %     for likelihood, choose normal (leads to MSE) and fix variance

\subsection{Training}
    \subsubsection{Process}
        RL-agents normally train with a specific loss function, which will be
        abstracted by $Loss_{RL}$. As the RL-agent acts as the encoder, it
        is responsible for the bitrate. Therefore we add the loss from
        \ref{eq:BitRate} to the standard RL loss:
        \begin{equation}\label{eq:RL_Training_Loss}
            \mathbb{E}_{x}[Loss_{RL}(x) + \alpha\cdot \mathbb{E}_{u \sim U[-\epsilon, \epsilon]}[log(P(Z=\hat{z} + u))]]
        \end{equation} where $\alpha \in \mathbb{R}$ is chosen to balance the two terms.
        Both expectations will be approximated by empirical means over the training
        data. As the RL-agent is independent of the decoder, we can also train it
        separately first.

        After training the RL-agent, we fix its values. Therefore, at this point
        the encoding to the latents is fixed. Now the decoder is trained. For the decoder, we use the Loss
        schemes given in section \ref{sub:Distortion}.


    \subsubsection{Adaptive alpha (omitted)}\label{sub:Adaptive_Alpha}
        During training, an issue arose in choosing $\alpha$ (see Experiments and
        Evaluation) which we believe was caused by the RL reward: the agent behaves
        randomly in early iterations and earns little reward, but improves
        drastically once learning starts. Therefore, choosing $\alpha$ too large
        prevents the agent from ever learning to complete the task, while choosing
        $\alpha$ too small allows the agent to ignore the cross entropy loss during action
        fine-tuning. We tried to resolve this with a non-static $\alpha$: as
        the reward grows, the relative importance of the two terms should stay
        approximately the same. We designed several adaptation schemes (others can be
        found in Future Work), but the general chosen algorithm was to update $\alpha$
        each time the ratio threshold was violated. We implemented this by
        \begin{equation}
            \alpha = \alpha \cdot 10 \cdot e^{-i / 1000}
        \end{equation}
        where $i$ is the number of iterations. Multiplying by the negative iterations
        ensures that $\alpha$ changes most drastically at the beginning, but
        converges over time.
\section{Experiments and Evaluation}
\subsection{Environment}
We used the game "Breakout" from the Gym toolkit \cite{brockman2016openai} as an
environment. Breakout is a game in which the player controls a paddle to bounce the ball and destroy blocks. The player wins if no block is left, but loses if the ball is dropped.

The RL agent learning and playing the game was Proximal
Policy Optimization (PPO) \cite{raffin2019stable}. PPO is an on-policy
algorithm, that uses an actor critic approach. We started with the stablebaselines3 implementation \cite{raffin2019stable}, and
modified it to our needs.

\subsection{Dimension reduction}
Our first experiments featured the basic dimension reduction method,
without the modifications for compression. The goal was to verify the
required reconstruction information was present within the feature extractor's lower dimensional
representation. The qualitative results can be
seen in Figure \ref{fig:baseline_MSE} (Note: images came
from training set). \\

\begin{figure}[H]
    \centering
    \includegraphics[width=\linewidth]{images/orig_reconstructed0.0.png}
    \caption{Baseline method (no compression) with decoder trained on MSE for 10,000 iterations}
    \label{fig:baseline_MSE}
\end{figure}


\subsection{Compression}\label{sub:Results_Compression} Now we added the
compression loss to the RL agent training, equation \ref{eq:RL_Training_Loss}
and trained the agent with this loss function. We modified the $\alpha$ value,
looking at the performance of the decoder after some epochs. An $\alpha$ of
$1e-4$ worked best, therefore we chose this value for an extended run over $1e5$
epochs. Tested on an independently created test dataset, this resulted in an
entropy of 1.7 bits per latent dimension, when the latent values were rounded to
3 digits. The actual bitrate given by the cross-entropy \ref{eq:BitRate} is
higher however, as we must choose a fixed latent encoding and transmission
distribution. For training we chose a Normal distribution with mean 0 and
learned variance, also used for testing. The expected variance per dimension was
estimated by the mean per dimension over the whole test dataset. However, the
test cross-entropy was significantly higher, mostly in the order of at least
times 1000. A visual investigation of the latent values showed that while the
entropy of the latent values was low, some values deviated from 0 significantly
in the order of 100, leading to a probability of nearly 0.

After training the RL Agent, the decoder was trained for $2e5$ epochs. This
resulted in an MSE Loss of 1895. As discussed in \ref{sec:Introduction}, MSE
is an unideal metric for performance, since it gives each pixel the same
value. Therefore we investigated some images qualitatively. An example can be seen in figure
\ref{fig:final_agent}.
\begin{figure}[H]
    \centering
    \includegraphics[width=\linewidth]{images/orig_reconstructed_final_agent.png}
    \caption{Final agent}
    \label{fig:final_agent}
\end{figure}

\subsection{Latent Loss Scheme}
As discussed in section \ref{sec:Introduction}, $L_2$ Loss may be suboptimal and
we introduced a latent loss scheme \ref{sub:Distortion} as an alternative.
Training a decoder with the latent loss scheme showed an image quality decrease
compared to $L_2$ Loss, (see figure \ref{fig:baseline_MSE_latent}) (also from
the training set).

\begin{figure}[H]
    \centering
    \includegraphics[width=\linewidth]{images/orig_reconstructed_rl3.0.png}
    \caption{Baseline method (no compression) with decoder trained on MSE for 10,000 iterations, then latent MSE for 10,000 iterations}
    \label{fig:baseline_MSE_latent}
\end{figure}

\subsection{Adaptive Alpha}
When implementing our custom loss function with static $\alpha$, several
rudimentary tests were performed to verify the model behaved as expected. One
such test involved varying $\alpha$: we expected that a lower value would
prioritize task performance, while a higher value would prefer a lower bitrate.
However, we found this was not the case: there seemed to be just as much
variation between independent tests of the same $\alpha$ as changing $\alpha$.
Given our hypothesis for where the problem lay (see \ref{sub:Adaptive_Alpha}),
we developed the adaptive $\alpha$ scheme. However, initial results showed no
improvement and this was moved to Future Work.

\subsection{Pretrained agents}
Another idea was that adding the compression loss to the RL agent from the
beginning on was too large of a constraint for the agent to learn anything.
Therefore, we tried to fix this by pretraining the agent before adding
the compression loss to the agent. On the one hand, the additional compression
loss reduced the entropy but on the other hand, the decoder did not learn
after the training, so this procedure showed no improvements.



\section{Discussion}
As can be seen in Figure 1, our baseline with the image MSE produced encouraging results: the location of the paddle was consistently preserved. However, areas of improvement include the artifacts that exist in this paddle along with the loss of the ball. As seen in Figure 2, use of the latent loss scheme resulted in extreme artifacts, as well as the loss of the paddle and did not result in ball recovery. A new scheme or use of this scheme would be needed.    
\end{multicols*}

\bibliographystyle{abbrv}
\bibliography{tex/06_Literature.bib}

\end{document}
