\section{Introduction}\label{sec:Introduction}
Image compression exists in multiple forms. Lossless compression mandates
original data be perfectly reconstructed, but is limited by data entropy. To
compress further, information content must be sacrificed. Therefore, lossy
compression minimizes the loss for a specific bitrate. But which content should
be discarded? Early image compression loss-functions, like $L_2$-loss, give each
pixel equal value. However, some parts retain more valuable information than
others. Research has hand-crafted improved metrics, like MS-SSIM
~\cite{1292216}, but defining these is task-specific. Ultimately, achieving the
most extreme forms of compression requires task-dependent methods to quantify
information value, preserving only the minimum required to complete the task.
However, general methods for this process would be useful. This facilitates a
natural transition into the reinforcement learning (RL) space, where agents act
on an environment to maximize rewards. The goal of this research is to define a
method through which data is compressed such that the agent can maximize the
reward in the reconstructed environment.
